\textgreek{
    Παρακάτω θα περιγράψουμε την διεπαφή του χρήστη καθώς και τον τρόπο
    με τον οποίο θα γίνεται η αναζήτηση.
}

\subsection{\textgreek{Εφαρμογή αναζήτησης}}
\textgreek{
    Για την χρήση του συστήματος απο τελικούς χρήστες θα υλοποιηθεί διαδυκτιακή
    εφαρμογή. Πιο συγκεκριμένα, θα υπάρχει ένας διακομιστής ο οποίος θα δέχεται
    αιτήματα αναζήτησης και θα απαντάει με τα αποτελέσματα σε μορφή
    $json$ έπειτα τα αποτελέσματα θα εμφανίζονται στον φυλλομετρητή του χρήστη.
}

\subsection{\textgreek{Φίλτρα αναζήτησης}}
\textgreek{
    Ο χρήστης θα έχει την δυνατότητα να επιλέξει με βάση ποιό πεδίο θέλει να
    γίνει η αναζήτηση (επιχειρήσεις, κριτικές, υποδείξεις) καθώς και την δυνατότητα
    αναζήτησης χωρίς τον καθορισμό κάποιου πεδίου.

    Κατά την αναζήτηση χωρίς καθορισμό πεδίου το σύστημα θα πρέπει να είναι σε
    θέση να δίνει βαρύτητα στα πεδία και να επιστρέφει τα πιο σχετικά αποτελέσματα.
}

\subsection{\textgreek{Έλεγχος σφαλμάτων}}
\textgreek{
    Το σύστημα επιπλέον θα υποστηρίζει έλεγχο σφαλμάτων καθώς και δυνατότητα
    αυτόματης συμπλήρωσης του ερωτήματος. Έτσι για παράδειγμα εάν ο χρήστης κάνει
    αναζήτηση για επιχειρήσεις με όνομα $Ssame$ τότε το σύστημα θα του προβάλει ερώτηση
    για το έαν εννοούσε $Sesame$.
}

\subsection{\textgreek{Αναδιάταξη αποτελεσμάτων}}
\textgreek{
    Το σύστημα θα υποστηρίζει αναδιάταξη αποτελεσμάτων με βάση τα πεδία της
    επιχείρησης. Αυτό θα γίνεται χρησιμοποιώντας ευρετήριο της βάσης δεδομένων
    στα πεδία που θέλουμε να υποστηρίζουν αναδιάταξη.
}
\subsection{\textgreek{Εύρεση αποτελεσμάτων}}
\textgreek{
    Το σύστημα ψάχνει στα ευρετήρια που έχει δημιουργήσει για την εύρεση των επιχειρήσεων
    που περιέχουν την λέξη αναζήτησης, $query$. Βρίσκει το μοναδικό $business\_id$ και συντάσσει
    ένα $query$ για την βάση δεδομένων μας, η οποία και επιστρέφει τα δεδομένα για το ζητούμενο
    ερώτημα.
}
