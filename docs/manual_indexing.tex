

\lstset{style=mystyle}

\subsection{\textgreek{Δημιουργία ευρετηρίων}}
\paragraph{\textgreek{Έγγραφα}}\textgreek{
Χρησιμοποιήσαμε τις μεθόδους της $lucene$ για την δημιουργία των εγγράφων.
Συγκεκριμένα ένα εγγράφο για κάθε επιχείρηση με πεδία, $business\_id$, $business\_name$, $business\_categories$,
$review\_id$, $review\_text$, $tip\_id$, $tip\_text$.
Δηλαδή σε κάθε έγγραφο αποθηκεύουμε όλη την πληροφορία που θα χρειαστούμε για κάθε επιχείρηση.
}

\begin{lstlisting}[language=Java]

private void createBusinessIndex(String city) throws SQLException, IOException {
    Document docEntry;
    Path path = Paths.get(indexDir, "businesses");
    Directory businessIndex = FSDirectory.open(path);
    IndexWriterConfig indexWriterConfig = new IndexWriterConfig(analyzer);
    indexWriterConfig.setOpenMode(IndexWriterConfig.OpenMode.CREATE);
    IndexWriter indexWriter = new IndexWriter(businessIndex, indexWriterConfig);

    String sqlQuery = sqlBusinessesIdxColsOfCity(city);
    List<String> columns = Arrays.asList("id", "name", "categories");

    PreparedStatement pst = dbConnection.prepareStatement(sqlQuery);
    pst.setFetchSize(100);
    ResultSet rs = pst.executeQuery();

    long startTime = System.currentTimeMillis();
    int cnt = 0;

    System.out.println(">>> Starting indexing");
    while (rs.next()) {
        docEntry = new Document();
        for (int i = 0; i < columns.size(); i++) {
            docEntry.add(new Field(columns.get(i), rs.getString(i + 1), TextField.TYPE_STORED));
        }

        String sqlReviews = Shortcuts.sqlReviewsIdxColsWhereBusinessIdIs(rs.getString(1));
        PreparedStatement pstRevs = dbConnection.prepareStatement(sqlReviews);
        pstRevs.setFetchSize(100);
        ResultSet rsRevs = pstRevs.executeQuery();
        while (rsRevs.next()) {
            docEntry.add(new Field("review_id", rsRevs.getString(1), TextField.TYPE_STORED));
            docEntry.add(new Field("review", rsRevs.getString(2), TextField.TYPE_STORED));
        }

        String sqlTips = Shortcuts.sqlTipsIdxColsWhereBusinessIdIs(rs.getString(1));
        PreparedStatement pstTips = dbConnection.prepareStatement(sqlTips);
        pstTips.setFetchSize(100);
        ResultSet rsTips = pstTips.executeQuery();
        while (rsTips.next()) {
            docEntry.add(new Field("tip_id", rsTips.getString(1), TextField.TYPE_STORED));
            docEntry.add(new Field("tip", rsTips.getString(2), TextField.TYPE_STORED));
        }

        indexWriter.addDocument(docEntry);
        cnt++;

        if (cnt % 100 == 0) {
            double elapsedTimeSec = (System.currentTimeMillis() - startTime) / 1000.0;
            System.out.printf("\tindexed %d businesses in %.2fsec\n", cnt, elapsedTimeSec);
        }
    }

    indexWriter.close();
    double elapsedTimeSec = (System.currentTimeMillis() - startTime) / 1000.0;
    System.out.printf("\tindexed %d businesses in %.2fsec\n", cnt, elapsedTimeSec);

\end{lstlisting}

\textgreek{
Όπως βλέπουμε στο παραπάνω κομμάτι κώδικα, όταν ο χρήστης ζητήσει την δημιουργία ευρετηρίων στέλνουμε ένα αίτημα
στην βάση δεδομένων το οποίο μας επιστρέφει τις πληροφορίες που χρειαζόμαστε για την κάθε επιχείρηση. Σε κάθε επανάληψη
δημιουργούμε το εγγραφο και τα πεδία και τα εισάγουμε στη στο ευρετήριο μας με την εντολή
}
\lstset{style=secondStyle}
\begin{lstlisting}
indexWriter.addDocument(docEntry);
\end{lstlisting}

\subsection{SpellChecker}
\textgreek{
Υλοποιήσαμε την πρόβλεψη λέξεων και την διόρθωση λαθών. Χρησιμοποιήσαμε την κλάσση $Spellchecker$ της $lucene$ για να
δημιουργήσουμε ευρετήριο στο οποίο θα βρούμε τις πιο συναφείς λέξεις για να προτείνουμε στον χρήστη. Η συνάφεια
μετριέται με την χρήση της $Levenstein$ απόστασης.
}
\lstset{style=mystyle}
\begin{lstlisting}

\end{lstlisting}
