\subsection{\textgreek{Επεξεργασία Ερωτημάτων}}
\textgreek {
    Ο διακομιστής στέλνει αίτημα αναζήτησης και καλεί την μέθοδο $search$ στο αρχείο $LuceneWrapper.java$.
    Η μέθοδος αυτή επεξεργάζεται το κείμενο του ερωτήματος και εντοπίζει αν είναι ερώτημα ενός ή πολλαπλών όρων.
    Αν είναι ένας όρος, κάνει αναζήτηση σε όλα τα πεδία του ευρετηρίου (δηλ. όνομα, κατηγορίες, κριτικές, υποδείξεις)
    δημιουργώντας έναν $multiFieldParser$.
    Αν πάλι το ερωτημά μας περιέχει ειδικούς χαρακτήρες και εντολές δημιουργείται $BooleanParser$ ή $PhraseParser$ για
    δυαδικές αναζητήσεις και αναζήτηση ολόκληρης φράσης, λέξης κλειδί αντίστοιχα.
    Η κλάσση $QueryParser$ επιτρέπει επίσης $wildcard$ αναζητήσεις και $term$ $boosting$.
    Στο ερώτημα μας, όπως και στην δημιουργία του ευρετηρίου χρησιμοποιήσαμε $StandardAnalyzer$ με την προσθήκη των 
    $default$ $stopWords$ που έχει η $lucene$.
}

\begin{lstlisting}[language=Java]
public List<SearchResult> search(String queryText, int resultsNum, String orderBy)
throws ParseException, SQLException, IOException, InvalidTokenOffsetsException {

    List<SearchResult> results = new ArrayList<>();
    List<String> fields = Arrays.asList("name", "categories", "review", "tip");
    QueryParser queryParser;
    IndexReader businessIndexReader = DirectoryReader.open(businessIndex);
    IndexSearcher searcher = new IndexSearcher(businessIndexReader);
    String[] querySplit = queryText.split(":");

    if (querySplit.length == 1) {
        queryText = querySplit[0];
        queryParser = new MultiFieldQueryParser(fields.toArray(new String[fields.size()]), analyzer);
    } else {
        if (!querySplit[1].matches("\\s+")) {
            querySplit[0] = querySplit[0].replaceAll("\\s+", "");
            if (!fields.contains(querySplit[0])) {
                queryText = "";
                for (String s : querySplit) {
                   queryText = queryText + " " + s;
                }
                queryText = queryText + "";
                queryParser = new MultiFieldQueryParser(fields.toArray(new String[fields.size()]), analyzer);
            } else {
                queryParser = new QueryParser(querySplit[0], new StandardAnalyzer(an.getStopWordSet()));
            }
        } else {
            querySplit = queryText.replaceAll("\\s+", "").split(":");
            queryText = querySplit[0];
            queryParser = new MultiFieldQueryParser(fields.toArray(new String[fields.size()]), analyzer);
        }
    }

    Query query = queryParser.parse(queryText);
    System.out.println("Lucene Query: " + query.toString());

    TopDocs topDocs = searcher.search(query, resultsNum);
    List<String> businessIDs = new ArrayList<>();
    HashMap<String, String> businessHighlight = new HashMap<>();

    for (ScoreDoc top : topDocs.scoreDocs) {
        String businessID = searcher.doc(top.doc).get("id");
        String highlight = highLightQuery(query, top, queryParser.getField(), businessIndexReader, searcher);
        businessIDs.add(businessID);
        businessHighlight.put(businessID, highlight);
    }

    List<Business> businesses = Getters.businessesByIDs(dbConnection, businessIDs, orderBy);
    for (int i = 0; i < businesses.size(); i++) {
        Business business = businesses.get(i);
        String highlight = businessHighlight.get(business.getId());
        results.add(new SearchResult(business, highlight));
    }

    return results;
}

\end{lstlisting}

\subsection{\textgreek{Επιστροφή αποτελεσμάτων}}
\textgreek{
Τα $ids$ απο τα κορυφαία αποτελέσματα αποθηκεύονται σε μια λίστα αλφαριθμητικών, η οποία χρησιμοποιείται για να συνταχθεί
ενα $sql$ ερώτημα στην βάση μας και να επιστραφούν τα δεδομένα για την κάθε επιχείρηση.
}

\subsection{Highlighting}
\textgreek{
Με το πέρας της αναζήτησης χρησιμοποιούμε την κλάσση της $lucene$ $highlighter$ για να βρούμε και να
επισημάνουμε το κείμενο της αναζήτησης μας.
Η αναζήτηση αυτή επιστρέφει ένα $html formatted$ κείμενο, το οποίο το χρησιμοποιούμε στην προβολή των αποτελεσμάτων.
}

\subsection{\textgreek{Αναδιάταξη Αποτελεσμάτων}}
\textgreek{
Η μεθοδός μας παίρνει σαν παράμετρο τον τρόπο με τον οποίο θέλουμε να ταξινομήσουμε τα αποτελέσματά μας.
Η παράμετρος αυτή περνάει σαν όρισμα στην μέθοδο που συντάσσουμε το $sql$ ερώτημα και χρησιμοποιείται απο την βάση.
Η βάση αναδιατάσσει τα αποτελέσματα με βάση:
}
\begin{enumerate}
    \item\textgreek{ Με βάση τα αστέρια της κάθε επιχείρησης (με φθίνουσα ή αύξουσα σειρά)}
    \item\textgreek{ Με βάση το πλήθος των κριτικών της κάθε επιχείρησης (φθίνουσα ή αύξουσα σειρά)}
    \item\textgreek{ Με βάση το πλήθος των χρηστών που πάτησαν πάνω στην επιχείρηση σε προηγούμενες αναζητήσεις ($clickthrough$)}
    \item\textgreek{ Με βάση την προεπιλεγμένη κατάταξη της $lucene$ }
\end{enumerate}

\subsection{FAQs \& Seach history}
\textgreek{
Η εφαρμογή μας κρατάει ιστορικό των αναζητήσεων και των κλικ σε κάθε επιχείρηση. Πιο συγκεκριμένα, μετά την υποβολή μιας
ερώτησης το σύστημα κρατάει στην βάση δεδομένων στατιστικά για την ερώτηση (συχνότητα αναζήτησης) απο τις οποίες μπορεί
να προτείνει στον χρήστη εναλλακτικά ερωτήματα. Ο αριθμός των κλικ ανά επιχείρηση αποθηκεύεται και χρησιμοποιείται για
την αναδιάταξη των αποτελεσμάτων και πιθανώς για περαιτέρω μελέτη και ανάλυση.
}
