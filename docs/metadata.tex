\textgreek{
    Το σύστημα κατα την λειτουργία του θα κρατάει στατιστικά στοιχεία
    απο τις προηγούμενες αναζητήσεις που έχουν γίνει.

    Με βάση τα μετα-δεδομένα το σύστημα θα μπορεί να εμφανίσει τις πιο δημοφιλείς
    επιχειρήσεις, να αναδιατάξει τα αποτελέσματα με βάση τα στατιστικά ερωτήσεων,
    να υποδείξει στους κατόχους των επιχειρήσεων τις πιο σημαντικές λέξεις κλειδιά
    για την επιχείρηση τους και να τους βοηθήσει να βγάλουν χρήσιμα συμπεράσματα για αυτές.
    Τα στατιστικά λειτουργίας είναι ιδιαίτερα χρήσιμα για τον έλεγχο ορθής λειτουργίας
    του συστήματος καθώς και την σύγκριση του με άλλα παρόμοια συστήματα.

    \paragraph{Στατιστικά ερωτήσεων}
    Το σύστημα θα κρατάει στατιστικά για της ερωτήσεις που έχουν γίνει
    καθώς και για τις πιο δημοφιλείς επιχειρήσεις.

    \paragraph{Στατιστικά εμφάνισης}
    Το σύστημα θα κρατάει στατιστικά στοιχεία για τον αριθμό εμφανίσεων
    των επιχειρήσεων στα αποτελέσματα, την θέση της επιχείρησης στα αποτελέσματα
    καθώς και την ερώτηση που την έφερε σε αυτή την θέση.

    \paragraph{Στατιστικά λειτουργίας}
    Το συστημα θα κρατάει πληροφορία για τον χρόνο εξυπηρέτησης τον ερωτημάτων,
    τον αριθμό των αιτημάτων που εξυπηρέτησε, κτλπ...

    \paragraph{Χρήση Στατιστικών}
    Με την χρήση των παραπάνω στατιστικών θα μπορούμε να υλοποιήσουμε λειτουργίες όπως $FAQ$, $ranking$
    των αποτελεσμάτων με βαση το $clickThrough$ και με βάση το ιστορικό.


}
